%%%%%%%%%%%%%%%%%%%%%%%%%%%%%%%%%%%%%%%%%%%%%%%%%%%%%%%%%%%%%%%%%%%%%%%%%%%
% manual.tex
% Reference manual for bwti (LaTeX source)
% Current version for bwti is 3.0
% Created by: Abhijit Das (Barda) IISc Bangalore
% Last modified: Nov 28 1998
%%%%%%%%%%%%%%%%%%%%%%%%%%%%%%%%%%%%%%%%%%%%%%%%%%%%%%%%%%%%%%%%%%%%%%%%%%%
%\documentstyle[twoside,11pt]{article}
\documentstyle[twoside,11pt,times]{article}
\textheight = 9.6in
\textwidth = 6.5in
\topmargin = -0.5in
\oddsidemargin = 0.0in
\evensidemargin = -0.3in
\parskip 0.1in
\pagestyle{myheadings}
\markboth{\sl Manual for the \tt bwti \sl package}{\sl Author: Abhijit Das (Barda)}
\raggedbottom

\input bengali.sty

\def\bwti{{\tt bwti}}
\def\METAFONT{{\sf METAFONT}}
\def\newsect#1{\bigskip\pagebreak[2]\par\noindent{\LARGE\sf #1}\hfill\nopagebreak\medskip\nopagebreak\break\nopagebreak}
\def\bs{\char`\\}

\begin{document}
\thispagestyle{plain}
\centerline{\huge \bf Manual for the {\LARGE\tt bwti} package}\medskip
\centerline{\LARGE \bf Author: \sc Abhijit Das (Barda)}\bigskip
\hrule\bigskip

\noindent
Welcome to the \bwti\ package. \bwti\ stands for ``Bengali Writer
\TeX\ Interface''.\footnote{This name comes from the fact that
I designed this as an extension of my previous endeavor named
``Bengali writer''. Point your web browser at
{\tt http://144.16.67.96/\char`\~barda/bengali} for details.}
This package is intended to help you typeset
in \TeX\ or \LaTeX\ using a Bengali font. This package consists of
\METAFONT\ files that define the Bengali characters and a style
file that defines several easy-to-use rules for typesetting using
these font files. This manual explains how to install the
\bwti\ package in your machine and how to use the style file for
typesetting in Bengali.

\newsect{How to install \bwti}
First uncompress the file and retrieve the components from the archive\\
{\tt \%}gunzip bwti.tar.gz\\
{\tt \%}tar xvf bwti.tar\\
(If you do not have {\tt gunzip}, get this software (free) from GNU.)
You will see that four new directories have been created. The
{\tt src} directory contains the \METAFONT\ source files. Put all these
\METAFONT\ files in a place where your \TeX\ or \LaTeX\ program will search for
\METAFONT\ source files. A typical example for a unix machine is
{\tt /usr/lib/texmf/fonts/tmp/src}. You should have permission to write in this
directory. Otherwise consult your system administrator. After you install
the source files you may put the files in the {\tt tfm} directory
in an appropriate place where your system searches for {\tt tfm} (\TeX\ font
metric) files,
for example, in {\tt /usr/lib/texmf/fonts/tmp/tfm} (Note that this is not
a must). This completes the installation
of the \METAFONT\ files.
Note that the {\tt pk} files are not provided. Your system should automatically create
these files when you view your \TeX\ or \LaTeX\ documents that contain the above
fonts.

The directory {\tt sty} contains a style file
named {\tt bengali.sty} that you should include in
your document where you want to include Bengali texts. You may do this by
a simple {\tt \char`\\input} command. Alternatively you may put
this style file in a directory where \LaTeX\ looks for style files and then
include the file as \\ 
{\tt \char`\\documentstyle\char`\[10pt,bengali\char`\]\char`\{article\char`\}}.

Finally the directory {\tt doc} contains a reference manual for the \bwti\
package. Try compiling the manual using \LaTeX. If it
compiles without any error message (neglect a few overfull hbox or vbox
warnings), view it and if you get the results as
you find in the file {\tt manual.ps},
your installation is complete.

\bigskip
The rest of this manual is devoted to a discussion on how you can use
the file {\tt bengali.sty} for typesetting in Bengali. However you can
completely avoid using this style file if you know the exact ASCII
mappings for all the Bengali characters namely vowels, consonants, vowel
forms, conjunct consonants and so on (See the tables at pages 7 and 8).
This is however inconvenient in that
in many cases this mapping is counter-intuitive
(particularly in the case of conjunct consonants). {\tt bengali.sty},
on the other hand, gives you a high-level view for treating
different characters uniformly. More specifically, {\tt bengali.sty}
defines several commands to typeset the Bengali characters, that ``sound''
like the original characters when pronounced. Also
{\tt bengali.sty} provides simple commands for changing the font size.
Let us take a simple example. Suppose you want to typeset the Bengali
word {\bn \sh\kt}. Without using the {\tt bengali.sty} file, you may do
this as follows:
\begin{verbatim}
\font\myfont=bnr10
{\myfont S\char130}
\end{verbatim}
The letter {\bn S} is defined in the file {\tt bnr10.mf} as the
letter ``S'' and the conjunct consonant {\bn \kt} is defined
as the letter with ASCII value 130. On the other hand, if you use
{\tt bengali.sty}, you may simply typeset this as
{\tt \{\char`\\bn \char`\\sh\char`\\kt\}} or as
{\tt \{\char`\\bn S\char`\\kt\}}.

Note however that you still have to remember a few mappings, particularly
those for the vowels and consonants. In this manual you will get a complete
listing of all these mappings, that is, what to type to get the desired
characters. A caution: using {\tt bengali.sty} may leave you
lost in a dense forest of backslashes and awkward space commands. We will discuss
these topics later. It's worth noting here only that there is no
straightforward and easy way of typesetting in an Indian language using
a system that understands only the ASCII type of character encoding.
{\tt bengali.sty} is designed as
a compromise (more technically, as a transliteration scheme) between them.

\newsect{The {\tt \char`\\bn} environment\footnote{Don't take the
word ``environment'' in the \LaTeX\ sense. I could have have defined a
\LaTeX\ environment called {\tt bn}, but all that is too \LaTeX nical.
\TeX\ users might get angry!! I want to make {\tt bengali.sty} work the
same way for \TeX\ and \LaTeX\ users.}}
{\tt bengali.sty} essentially provides the typeface changing command
{\tt \char`\\bn} with a few (actually, a lot)
macro definitions active inside the environment. Start typesetting
in Bengali by typing ``{\tt \char`\{\char`\\bn}''.\footnote{You may
omit the opening brace, but in that case you can never return to
typesetting English texts.
Inside the {\tt \bs bn} environment you may use the command {\tt \bs ENG} for
typesetting only in 10pt roman.}
This sets the font to the Bengali font of the {\tt bwti} package in
10pt. Note that whatever fontsize you define in your {\tt \bs documentstyle}
command (for \LaTeX\ users), the default fontsize for {\tt \bs bn} is
{\em always} 10pt. You may however change the font-size by a command like
{\tt \bs large} or {\tt \bs small}. We will discuss this later.
After you typeset your Bengali text, close {\tt \bs bn}'s scope
(more correctly, the local group) by typing ``\}''. This leaves you in
the environment that was active before the {\tt \bs bn} command.

In what follows we will assume that the {\tt \bs bn} environment is active
and discuss what you can precisely do (and what you can not) in this
environment.

\newsect{Changing font size}
The following font-sizes are provided.\footnote{This convention is
similar to that provided by \LaTeX. However the font sizes are close to
but not exactly the same as defined in \LaTeX. Note that the sizes
{\tt \bs footnotesize} and {\tt \bs scriptsize} are not defined.}
\begin{verbatim}
\normalsize     \large       \small
                \Large       \tiny
                \LARGE
                \huge
                \Huge
\end{verbatim}
These sizes are essentially the 10pt font magnified by
{\tt \bs magstep}$^{\tt n}$ for ${\tt n} = 0,1,2,3,4,5,-1,-2$, where
{\tt \bs magstep} is 1.2. To see how to use these commands, let us take a
simple example.

\centerline{\bn sA {\large er {\Large gA {\LARGE mA  pA} \dh A} in} sA}

\noindent
This can be typeset by:
\begin{verbatim}
sA {\large er {\Large gA {\LARGE mA  pA} \dh A} in} sA
\end{verbatim}
Note carefully the scope of each definition. Each opening brace defines a new
local group and when it closes, the local group that was active before the
opening brace takes over. Note also that the same output can be obtained by:
\begin{verbatim}
sA \large er \Large gA \LARGE mA  pA \Large \dh A \large in\normalsize\ sA
\end{verbatim}

\newsect{Getting slanted output}
Slanted output can be obtained by the {\tt \bs sl} command. The {\tt \bs sl} command
does not change the font size. You can change the font size
inside a ``slanted'' environment
by a command defined described in the previous section. Thus
{\tt \{\bs bn \bs sl \bs LARGE srgm\}} and {\tt \{\bs bn \bs LARGE \bs sl srgm\}}
both has the same effect, namely : {\bn \sl \LARGE srgm}. You can change
from the slanted style to the normal style by the {\tt \bs nsl} command. For
example {\tt \{\bs bn \bs Large \bs sl sr \bs nsl gm\}} outputs
{\bn \Large \sl sr \nsl gm}.

\newsect{Getting bold output}
No boldface fonts are defined. However, you may use the commands
{\tt\bs bnbold},{\tt\bs bnBold} and {\tt\bs bnBOLD}
defined in {\tt bengali.sty} to typeset text that
{\em looks like} one typeset in boldface.
Each of them takes one argument
and prints multiple copies\footnote{{\tt\bs bnbold} puts 16 copies,
{\tt\bs bnBold} 36 copies and {\tt\bs bnBOLD} 64 copies.} of the argument.
You should use different commands for different font sizes. The
recommended usage is tabulated below.

\begin{tabular}{cc}
\bf Command& \bf Font size\\
{\tt\bs bnbold}& {\tt\bs normalsize},{\tt\bs large}\\
{\tt\bs bnBold}& {\tt\bs Large},{\tt\bs LARGE}\\
{\tt\bs bnBOLD}& {\tt\bs huge},{\tt\bs Huge}\\
\end{tabular}

None of these commands is suitable for font sizes {\tt\bs small} and
{\tt\bs tiny}. Indeed you are recommended not to use them at
{\tt\bs normalsize} also. These commands are suitable for printing
headers at larger font sizes. See the section titled ``Sectioning''.
Another caution. The output of {\tt\bs bnbold} etc is set in an
{\tt\bs hbox} and hence {\em can not} span over multiple lines.

One final note. These bold-making commands are active outside the
{\tt\bs bn} environment also. Therefore, you may use them to typeset
English texts. For example, compare
\bnBold{\Large ABCD} obtained by {\tt\bs bnBold} with {\Large\bf ABCD}
obtained by {\tt\bs bf}.

\newsect{The alphabet}
The Bengali alphabet is defined in the \METAFONT\ files provided. The file
{\tt bengali.sty} defines how you can access the letters. A complete
listing of all these definitions can be found later in this manual.
Here it is worth mentioning a few general rules. The Bengali alphabet
consists of the following primitives: vowels ({\bn \a, \aa} etc.),
vowel forms ({\bn i\ ,\ I} etc.), consonants ({\bn k, K} etc.),
conjunct consonants ({\bn \kk, \nchh} etc.) and a few other special
symbols (like hasanta, ref etc.).
All vowel forms and consonants
can be printed by a single character. For example, {\bn i\ } is obtained by
typing {\tt i} and {\bn k} is obtained by typing {\tt k}. Some of the
consonants can also be printed by special commands. {\bn F}, for example,
can be printed either by {\tt F} or by {\tt \bs ss}. The vowels
like {\bn \A} or {\bn \i} can be produced by a backslash followed
by the code for the corresponding vowel forms ({\tt \bs A} or {\tt \bs i} etc.).
\footnote{This may seem rather counter-intuitive. A crude estimate tells that
the vowel forms are used 5 to 10 times more often than the bare vowels.
So this convention at least saves you some typing.}
You get the numbers {\bn 1234}$\cdots$ by simply typing 1234$\cdots$.
Special symbols like {\bn \jafala\hspace*{1em} \rafala\ \Ref} can be produced by commands
{\tt \bs jafala \bs rafala \bs Ref} etc. Finally all conjunct consonants
are typeset by special commands which in most cases consist of the codes
of the component consonants concatenated in the order in which the
component consonants occur in the conjunct consonant. For example, {\tt k} and
{\tt t} outputs {\bn k} and {\bn t} respectively and {\tt \bs kt} outputs {\bn \kt}.
The examples given later will help you understand all
these subtleties. You will nevertheless have to memorize the mapping tables
(or at least some parts of them) and once this is done (or if you keep
this manual just beside your computer keyboard), utilizing all the features
of the {\tt bwti} package should be easy.

\newsect{A note on spacing}
Suppose you want to typeset {\bn \e k}. You read from the mapping tables
that the codes for {\bn \e} and {\bn k} are respectively
{\tt \bs e} and {\tt k}. If you now write {\tt \bs ek} and
compile it, your system will halt with an error message
{\tt Undefined control sequence}. This is because the command {\tt \bs ek}
is not defined in the {\tt \bs bn} environment. You should type
{\tt \bs e k} instead to get the desired result. You need not worry about
the extra space you provide, because this space tells \TeX\ that {\tt \bs e}
is a command and this is followed by {\tt k}. Any number of spaces will
in fact have the same effect in this regard. If you really want to avoid
the space (may be for the sake of clarity), you may type
{\tt \{\bs e\}k} to get the same output.

So far so good !! But what will you do when you do really want the space?
For example, suppose that you want to write {\bn du\i\ itn}.
If you write {\tt du\bs i itn}, you will get
{\bn du\i itn}. In this case you should explicitly produce the
inter-word spacing by typing {\tt  du\bs i\bs\ itn}.
Note that backslash followed by a space tells \TeX\ to typeset an
inter-word spacing. If you have doubts, consult the \TeX\ or \LaTeX\
manual for clarification. Again you may avoid using the
spaces and space-commands by typing
{\tt du\{\bs i\} itn}.

One last comment. Note that {\bn e} is typeset {\em before} the consonant
whose sound it modifies, whereas {\bn A} is typeset {\em after} the
consonant. Thus you get {\bn epn} by typing {\tt epn}, whereas
{\tt pen} prints {\bn pen} though it
tallies with the English pronunciation.

All these may seem quite confusing (and embarrassing) to you. But as I claimed
earlier, it is not an easy task to teach \TeX\ an Indian language. We will
discuss about ligature tables and some other related issues in a later section.

\newsect{The {\tt \bs ENG} command}
The {\tt \bs ENG} command can be used to typeset English text in
10pt roman font. You can not change the font size and style in this
environment. Get out of the {\tt \bs bn} environment and use
\TeX's commands to typeset English texts. The {\tt \bs rm} command
works similarly except that it adjusts to the font size that was active
before the {\tt \bs bn} command.

\newsect{Footnotes}
The {\tt \bs footnote} command works differently in \TeX\ and \LaTeX.
\TeX\ prints the footnote in the current font, whereas \LaTeX\ prints
the footnote in a roman font of predefined size. You must invoke the
{\tt \bs bn} command within the argument of {\tt \bs footnote} for
printing Bengali footnotes if you use \LaTeX.

\newsect{Sectioning}
You may have serious
problems if you use the sectioning commands provided by \LaTeX\
inside the {\tt \bs bn} environment (or the {\tt \bs bn} command inside
the sectioning command argument). This problem can be
bypassed by redefining the sectioning commands. {\tt bengali.sty} provides
you with nine pre-defined heading commands, each of which takes one input and
prints it in a {\em single line}
using {\tt\bs bnbold}, {\tt\bs bnBold} or {\tt\bs bnBOLD}.
The commands also specify how you should justify your headings.
These commands are

\begin{tabular}{ll}
\multicolumn{1}{c}{\bf Commands}& \multicolumn{1}{c}{\bf Justification}\\\hline
{\tt\bs lbnh}, {\tt\bs Lbnh}, {\tt\bs LBNH}& left-justified heading\\
{\tt\bs cbnh}, {\tt\bs Cbnh}, {\tt\bs CBNH}& centered heading\\
{\tt\bs rbnh}, {\tt\bs Rbnh}, {\tt\bs RBNH}& right-justified heading\\
\end{tabular}
\hfill
\begin{tabular}{ll}
\multicolumn{1}{c}{\bf Commands}& \multicolumn{1}{c}{\bf Size}\\\hline
{\tt\bs lbnh}, {\tt\bs cbnh}, {\tt\bs rbnh}& {\tt\bs large}\\
{\tt\bs Lbnh}, {\tt\bs Cbnh}, {\tt\bs Rbnh}& {\tt\bs LARGE}\\
{\tt\bs LBNH}, {\tt\bs CBNH}, {\tt\bs RBNH}& {\tt\bs Huge}\\
\end{tabular}\hspace*{\parindent}

Note that none of these commands takes a multi-line argument. In addition,
they {\em do not} provide any automatic numbering facility. If many people
use \bwti\ and think that this is a limitation, I will fix it in a later
version.

\newsect{Mathematical formulas}
Mathematical formulas are to be typeset in the {\tt \bs bn} environment
exactly the way you do it outside. It is possible to
include Bengali text inside a mathematical formula, but with some effort.
If you really want to know how, try it out yourself as an {\em easy}
exercise. The point is that it is not very useful in practice. Here are
two examples for you.
$$\bn 23+\left[4-{(5-k^{\log(\mu\pm g_T)}_1)\over
                         \sqrt{\kh\div\hbox{\bn\nchh}}}\right]$$
$$23+\left[4-{(5-
  \hbox{\bn k}^{\log(\mu\pm\hbox{\bn\small g}_{\hbox{\bn\tiny T}})}_1)
  \over\sqrt{\hbox{\bn\kh}\div\hbox{\bn\nchh}}}\right]$$
And this is how they are typeset:
\begin{verbatim}
$$\bn 23+\left[4-{(5-k^{\log(\mu\pm g_T)}_1)\over
                         \sqrt{\kh\div\hbox{\bn\nchh}}}\right]$$
$$23+\left[4-{(5-
  \hbox{\bn k}^{\log(\mu\pm\hbox{\bn\small g}_{\hbox{\bn\tiny T}})}_1)
  \over\sqrt{\hbox{\bn\kh}\div\hbox{\bn\nchh}}}\right]$$
\end{verbatim}

\newsect{Ligatures}
I think it's not a lucrative option to introduce ligatures for
the Bengali alphabet (say, in particular, for conjunct consonants).
Firstly, if we want to define characters corresponding to every consonant-%
vowel combination (for example, for {\tt ka}, {\tt kA}, {\tt ki}, {\tt kI}
and so on, so that {\tt pen} typesets {\bn epn}), the number of
characters in the alphabet increases enormously. It is always
preferable to work with as few primitives as possible. Secondly,
ligatures like {\tt sh} for {\bn \sh} are often confusing. For example,
if you define {\bn s} by {\tt s} and {\bn h} by {\tt h} (this is certainly
the most intuitive choice), then it would be rather inconvenient
to write {\bn sh} because {\tt sh} will always produce {\bn \sh}. In
this case you will have to type {\tt s\{\}h} to get {\bn sh}. This
probably complicates things more than the convenience it is expected
to produce. \bwti, on the other hand, provides a few ligatures when
there is very little or no chance of confusion (for example, {\tt e}
produces {\bn e}, whereas {\tt ee} gives {\bn\ ee} -- you may not have
a situation where you want to typeset two successive {\bn e}'s)
or when it is absolutely necessary (for example, {\bn h + \R} does not
look like {\bn h{}\r} but like {\bn h\r}). Here goes a listing of all
ligatures provided by \bwti.

\halign{&\quad\bn#\hfil\cr
\noalign{\smallskip\noindent\bf Vowel forms\smallskip}
{\ENG a + a = A}&{\ENG  e + e = I}&{\ENG  i + i = I}&{\ENG  o + o = U}&{\ENG  u + u = U}&{\ENG  o + i = E}&{\ENG  o + u = O}\cr
\noalign{\smallskip\noindent\bf Special consonat-vowel combinations\smallskip}
g + u = gu& r + u = ru& r + U = rU& S + u = Su& h + u = hu& h + W = hW\cr
\noalign{\smallskip\noindent\bf Special conjunct consonants\smallskip}
k + \ \rf\ = k\rf& t + \ \rf\  = t\rf& {\tr\ + u = t\rf u}& {\dv\ + \ \rf\ = \dv\rf}& {\nt\ + u = \nt u}& {\nt\ + \ \rf\  = \nt\rf}& v + \ \rf\  = v\rf\cr
{\mv\ + \ \rf\ = \mvr}& {\lg\ + u = \lg u}& {\ssk\ + \ \rf\ = \ssk\rf}& {\sk\ + \ \rf\ = \sk\rf}& {\st\ + u = \st u}& {\st\ + \ \rf\ = \st\rf}\cr
\noalign{\smallskip\noindent\bf Punctuation symbols\smallskip}
` + ` = ``& ' + ' = ''& - + - = --& -- + - = ---& . + . = ..\cr
}

\newsect{What {\tt bwti} does not provide}
It's now time to concentrate on what you can not do with \bwti.
The font provided by \bwti\ is a light-weight font. You may have
unsatisfactory results with this font at a size smaller than 10pt.
You are recommended to use the {\tt \bs small} and in particular the
{\tt \bs tiny} fontsizes as seldom as possible.  \bwti\ does not provide
a boldface font (extended or nonextended) for the Bengali alphabet.
(See the section ``Getting bold output''.)
On the other hand, italics, fixed width (tt) and sans serif typestyles are not well
defined in Bengali typesetting.

Kerning information are absent in any of the \METAFONT\ files. This
is however not something that can be ignored. Sorry, I am not a professional
font-designer, nor does this package have anything to do with my work in my
institute. I have very little idea how typesetting in Bengali is actually done.
Any kerning or related decision has to be taken rather intuitively or
empirically. I didn't do it. If you are not comfortable with certain
inter-letter spacing, you change it by \TeX's {\tt \bs kern} command.

The Bengali vowel {\tt Li} is not defined. This character is almost never used
in Bengali. If you have an esoteric situation where you really want to print
it, do this with the Bengali digit {\bn 9}. While defining the conjunct
consonants I tried to be as exhaustive as possible. Every conjunct
consonant listed in {\bn ibd\jf AsAgr}'s {\bn bN\Ref pir\ch y}
has been defined. A few other conjunct consonants most of which are
used in foreign words are also defined. Possibility of omissions can not
however be completely ruled out. If you find any, please bring that to
my notice.

\newsect{Example 1}
Suppose you want to typeset Joule's law of heating (in the old form where
the unit of heat is calorie).
{\bn
\Lbnh{juelr sUt\rf}
\e kiT pirbAhIet \u\t pAidt taaepr pirmAN ($H$) \u haaet p\rf baaiht
tiR\t\ ($i$)-\e r bg\Ref, pirbAhIr erA\dh\ ($R$) \e b\anuswar\
tiR\t\ p\rf bAehr smy ($t$)-\e r guNfelr siht smaanupaaitk.
\a\th\Ref aa\t,
$$H = {\rm J}i^2Rt$$
e\j\kh aaen {\rm J} \e kiT \dh\rf ubk. {\ENG J} {\sl juelr \dh\rf ubk}
naaem piri\ch t.
}

\noindent
Type the following to get it.
\begin{verbatim}
{\bn
\Lbnh{juelr sUt\rf}
\e kiT pirbAhIet \u\t pAidt taaepr pirmAN ($H$) \u haaet p\rf baaiht
tiR\t\ ($i$)-\e r bg\Ref, pirbAhIr erA\dh\ ($R$) \e b\anuswar\
tiR\t\ p\rf bAehr smy ($t$)-\e r guNfelr siht smaanupaaitk.
\a\th\Ref aa\t,
$$H = {\rm J}i^2Rt$$
e\j\kh aaen {\rm J} \e kiT \dh\rf ubk. {\ENG J} {\sl juelr \dh\rf ubk}
naaem piri\ch t.
}
\end{verbatim}

\newsect{The font files}
The following tables list the coding of the Bengali alphabet. The file
{\tt bnr10.mf} defines the complete Bengali alphabet, and
{\tt bnsl10.mf} defines the slanted version.
To get the ASCII code for a particular character just add the headings
of the row and the column to which the character belongs. For example,
the consonant {\bn \chh} is defined in file {\tt bnr10.mf} and is assigned
the ASCII value $64 + 3 = 67$ (which is the value for ``C'' in the ASCII
roman alphabet).

\def\showfont#1{%
\centerline{\vbox{\offinterlineskip
\halign{\strut\hfil\bf## && \vrule\enskip\hfil{\newfont{\char##}}\hfil\enskip\cr
&\multispan{16}\hfil#1\hfil\cr
\noalign{\smallskip}
& \omit\bf\hfil0\hfil& \omit\bf\hfil1\hfil& \omit\bf\hfil2\hfil& \omit\bf\hfil3\hfil
& \omit\bf\hfil4\hfil& \omit\bf\hfil5\hfil& \omit\bf\hfil6\hfil& \omit\bf\hfil7\hfil
& \omit\bf\hfil8\hfil& \omit\bf\hfil9\hfil& \omit\bf\hfil10\hfil& \omit\bf\hfil11\hfil
& \omit\bf\hfil12\hfil& \omit\bf\hfil13\hfil& \omit\bf\hfil14\hfil& \omit\bf\hfil15\hfil\cr
\omit\hfil& \multispan{16}\hrulefill\cr
0& 0& 1& 2& 3& 4& 5& 6& 7& 8& 9& 10& 11& 12& 13& 14& \omit\vrule\enspace\hfil\newfont\char15\hfil\enspace\vrule\cr
\omit\hfil& \multispan{16}\hrulefill\cr
16& 16& 17& 18& 19& 20& 21& 22& 23& 24& 25& 26& 27& 28& 29& 30& \omit\vrule\enspace\hfil\newfont\char31\hfil\enspace\vrule\cr
\omit\hfil& \multispan{16}\hrulefill\cr
32& 32& 33& 34& 35& 36& 37& 38& 39& 40& 41& 42& 43& 44& 45& 46& \omit\vrule\enspace\hfil\newfont\char47\hfil\enspace\vrule\cr
\omit\hfil& \multispan{16}\hrulefill\cr
48& 48& 49& 50& 51& 52& 53& 54& 55& 56& 57& 58& 59& 60& 61& 62& \omit\vrule\enspace\hfil\newfont\char63\hfil\enspace\vrule\cr
\omit\hfil& \multispan{16}\hrulefill\cr
64& 64& 65& 66& 67& 68& 69& 70& 71& 72& 73& 74& 75& 76& 77& 78& \omit\vrule\enspace\hfil\newfont\char79\hfil\enspace\vrule\cr
\omit\hfil& \multispan{16}\hrulefill\cr
80& 80& 81& 82& 83& 84& 85& 86& 87& 88& 89& 90& 91& 92& 93& 94& \omit\vrule\enspace\hfil\newfont\char95\hfil\enspace\vrule\cr
\omit\hfil& \multispan{16}\hrulefill\cr
96& 96& 97& 98& 99& 100& 101& 102& 103& 104& 105& 106& 107& 108& 109& 110& \omit\vrule\enspace\hfil\newfont\char111\hfil\enspace\vrule\cr
\omit\hfil& \multispan{16}\hrulefill\cr
112& 112& 113& 114& 115& 116& 117& 118& 119& 120& 121& 122& 123& 124& 125& 126& \omit\vrule\enspace\hfil\newfont\char127\hfil\enspace\vrule\cr
\omit\hfil& \multispan{16}\hrulefill\cr
128& 128& 129& 130& 131& 132& 133& 134& 135& 136& 137& 138& 139& 140& 141& 142& \omit\vrule\enspace\hfil\newfont\char143\hfil\enspace\vrule\cr
\omit\hfil& \multispan{16}\hrulefill\cr
144& 144& 145& 146& 147& 148& 149& 150& 151& 152& 153& 154& 155& 156& 157& 158& \omit\vrule\enspace\hfil\newfont\char159\hfil\enspace\vrule\cr
\omit\hfil& \multispan{16}\hrulefill\cr
160& 160& 161& 162& 163& 164& 165& 166& 167& 168& 169& 170& 171& 172& 173& 174& \omit\vrule\enspace\hfil\newfont\char175\hfil\enspace\vrule\cr
\omit\hfil& \multispan{16}\hrulefill\cr
176& 176& 177& 178& 179& 180& 181& 182& 183& 184& 185& 186& 187& 188& 189& 190& \omit\vrule\enspace\hfil\newfont\char191\hfil\enspace\vrule\cr
\omit\hfil& \multispan{16}\hrulefill\cr
192& 192& 193& 194& 195& 196& 197& 198& 199& 200& 201& 202& 203& 204& 205& 206& \omit\vrule\enspace\hfil\newfont\char207\hfil\enspace\vrule\cr
\omit\hfil& \multispan{16}\hrulefill\cr
208& 208& 209& 210& 211& 212& 213& 214& 215& 216& 217& 218& 219& 220& 221& 222& \omit\vrule\enspace\hfil\newfont\char223\hfil\enspace\vrule\cr
\omit\hfil& \multispan{16}\hrulefill\cr
224& 224& 225& 226& 227& 228& 229& 230& 231& 232& 233& 234& 235& 236& 237& 238& \omit\vrule\enspace\hfil\newfont\char239\hfil\enspace\vrule\cr
\omit\hfil& \multispan{16}\hrulefill\cr
240& 240& 241& 242& 243& 244& 245& 246& 247& 248& 249& 250& 251& 252& 253& 254& \omit\vrule\enspace\hfil\newfont\char255\hfil\enspace\vrule\cr
\omit\hfil& \multispan{16}\hrulefill\cr
}}}}

\def\showboldfont#1{%
\centerline{\vbox{\offinterlineskip
\halign{\strut\hfil\bf## && \vrule\enskip\hfil\bn\bnbold{\char##}\hfil\enskip\cr
&\multispan{16}\hfil#1\hfil\cr
\noalign{\smallskip}
& \omit\bf\hfil0\hfil& \omit\bf\hfil1\hfil& \omit\bf\hfil2\hfil& \omit\bf\hfil3\hfil
& \omit\bf\hfil4\hfil& \omit\bf\hfil5\hfil& \omit\bf\hfil6\hfil& \omit\bf\hfil7\hfil
& \omit\bf\hfil8\hfil& \omit\bf\hfil9\hfil& \omit\bf\hfil10\hfil& \omit\bf\hfil11\hfil
& \omit\bf\hfil12\hfil& \omit\bf\hfil13\hfil& \omit\bf\hfil14\hfil& \omit\bf\hfil15\hfil\cr
\omit\hfil& \multispan{16}\hrulefill\cr
0& 0& 1& 2& 3& 4& 5& 6& 7& 8& 9& 10& 11& 12& 13& 14& \omit\vrule\enspace\hfil\bn\bnbold{\char15}\hfil\enspace\vrule\cr
\omit\hfil& \multispan{16}\hrulefill\cr
16& 16& 17& 18& 19& 20& 21& 22& 23& 24& 25& 26& 27& 28& 29& 30& \omit\vrule\enspace\hfil\bn\bnbold{\char31}\hfil\enspace\vrule\cr
\omit\hfil& \multispan{16}\hrulefill\cr
32& 32& 33& 34& 35& 36& 37& 38& 39& 40& 41& 42& 43& 44& 45& 46& \omit\vrule\enspace\hfil\bn\bnbold{\char47}\hfil\enspace\vrule\cr
\omit\hfil& \multispan{16}\hrulefill\cr
48& 48& 49& 50& 51& 52& 53& 54& 55& 56& 57& 58& 59& 60& 61& 62& \omit\vrule\enspace\hfil\bn\bnbold{\char63}\hfil\enspace\vrule\cr
\omit\hfil& \multispan{16}\hrulefill\cr
64& 64& 65& 66& 67& 68& 69& 70& 71& 72& 73& 74& 75& 76& 77& 78& \omit\vrule\enspace\hfil\bn\bnbold{\char79}\hfil\enspace\vrule\cr
\omit\hfil& \multispan{16}\hrulefill\cr
80& 80& 81& 82& 83& 84& 85& 86& 87& 88& 89& 90& 91& 92& 93& 94& \omit\vrule\enspace\hfil\bn\bnbold{\char95}\hfil\enspace\vrule\cr
\omit\hfil& \multispan{16}\hrulefill\cr
96& 96& 97& 98& 99& 100& 101& 102& 103& 104& 105& 106& 107& 108& 109& 110& \omit\vrule\enspace\hfil\bn\bnbold{\char111}\hfil\enspace\vrule\cr
\omit\hfil& \multispan{16}\hrulefill\cr
112& 112& 113& 114& 115& 116& 117& 118& 119& 120& 121& 122& 123& 124& 125& 126& \omit\vrule\enspace\hfil\bn\bnbold{\char127}\hfil\enspace\vrule\cr
\omit\hfil& \multispan{16}\hrulefill\cr
128& 128& 129& 130& 131& 132& 133& 134& 135& 136& 137& 138& 139& 140& 141& 142& \omit\vrule\enspace\hfil\bn\bnbold{\char143}\hfil\enspace\vrule\cr
\omit\hfil& \multispan{16}\hrulefill\cr
144& 144& 145& 146& 147& 148& 149& 150& 151& 152& 153& 154& 155& 156& 157& 158& \omit\vrule\enspace\hfil\bn\bnbold{\char159}\hfil\enspace\vrule\cr
\omit\hfil& \multispan{16}\hrulefill\cr
160& 160& 161& 162& 163& 164& 165& 166& 167& 168& 169& 170& 171& 172& 173& 174& \omit\vrule\enspace\hfil\bn\bnbold{\char175}\hfil\enspace\vrule\cr
\omit\hfil& \multispan{16}\hrulefill\cr
176& 176& 177& 178& 179& 180& 181& 182& 183& 184& 185& 186& 187& 188& 189& 190& \omit\vrule\enspace\hfil\bn\bnbold{\char191}\hfil\enspace\vrule\cr
\omit\hfil& \multispan{16}\hrulefill\cr
192& 192& 193& 194& 195& 196& 197& 198& 199& 200& 201& 202& 203& 204& 205& 206& \omit\vrule\enspace\hfil\bn\bnbold{\char207}\hfil\enspace\vrule\cr
\omit\hfil& \multispan{16}\hrulefill\cr
208& 208& 209& 210& 211& 212& 213& 214& 215& 216& 217& 218& 219& 220& 221& 222& \omit\vrule\enspace\hfil\bn\bnbold{\char223}\hfil\enspace\vrule\cr
\omit\hfil& \multispan{16}\hrulefill\cr
224& 224& 225& 226& 227& 228& 229& 230& 231& 232& 233& 234& 235& 236& 237& 238& \omit\vrule\enspace\hfil\bn\bnbold{\char239}\hfil\enspace\vrule\cr
\omit\hfil& \multispan{16}\hrulefill\cr
240& 240& 241& 242& 243& 244& 245& 246& 247& 248& 249& 250& 251& 252& 253& 254& \omit\vrule\enspace\hfil\bn\bnbold{\char255}\hfil\enspace\vrule\cr
\omit\hfil& \multispan{16}\hrulefill\cr
}}}}

\mbox{}
\vfill
\def\newfont{\bn}
\showfont{\Large{\tt bnr10.mf}: The unslanted alphabet}

\vfill
\def\newfont{\bn\sl}
\showfont{\Large{\tt bnsl10.mf}: The slanted alphabet}
\vfill
\newpage

\mbox{}
\vfill
\showboldfont{\Large{\tt\bs bnbold} version of {\tt bnr10.mf}}
\vfill
\newpage

\newsect{Encoding of vowels and vowel forms}
\centerline{\begin{tabular}[t]{cccc}
vowel& code& vowel form& code\\\hline
{\bn \a}& {\tt \bs a} or {\tt a}& no sign&\\
{\bn \aa}& {\tt \bs A} or {\tt \bs aa} or {\tt aA}& {\bn aa}& {\tt A} or {\tt aa}\\
{\bn \AE}& {\tt \bs AE}& {\bn \ae}& {\tt\bs ae}\\
{\bn \i}& {\tt \bs i}& {\bn i}& {\tt i}\\
{\bn \I}& {\tt \bs I} or {\tt \bs ee} or {\tt\bs ii}& {\bn ee}& {\tt I} or {\tt ee} or {\tt ii}\\
{\bn \u}& {\tt \bs u}& {\bn u}& {\tt u}\\
\hline
\end{tabular}
\hfil
\begin{tabular}[t]{cccc}
vowel& code& vowel form& code\\\hline
{\bn \U}& {\tt \bs U} or {\tt \bs oo} or {\tt\bs uu}& {\bn oo}& {\tt U} or {\tt oo} or {\tt uu}\\
{\bn \R}& {\tt \bs R} or {\tt \bs Ri}& {\bn \r}& {\tt \bs r} or {\tt \bs ri} or {\tt W}\\
%{\bn 9}& {\tt 9}&&\\
{\bn \e}& {\tt \bs e}& {\bn e}& {\tt e}\\
{\bn \oi}& {\tt \bs E} or {\tt \bs oi}& {\bn E}& {\tt E} or {\tt oi}\\
{\bn \o}& {\tt \bs o}& no sign&\\
{\bn \ou}& {\tt \bs O} or {\tt \bs ou} or {\tt o}& {\bn O}& {\tt O} or {\tt ou}\\
\hline
\end{tabular}}

\newsect{Encoding of consonants}
\centerline{\begin{tabular}{cc|cc|cc|cc|cc}
cons. & code & cons. & code & cons. & code & cons. & code & cons. & code \\ \hline
{\bn k} & {\tt k} & {\bn \kh} & {\tt K} or {\tt \bs kh}&
          {\bn g} & {\tt g} & {\bn G} & {\tt G} or {\tt \bs gh} &
          {\bn \una} & {\tt q} or {\tt \bs una} \\
{\bn c} & {\tt c} or {\tt \bs ch} & {\bn \chh} & {\tt C} or {\tt \bs chh} &
          {\bn j} & {\tt j} & {\bn J} & {\tt J} or {\tt \bs jh} &
          {\bn Q} & {\tt Q} or {\tt \bs ina} \\
{\bn T} & {\tt T} & {\bn Z} & {\tt Z} or {\tt \bs Th} &
          {\bn D} & {\tt D} & {\bn X} & {\tt X} or {\tt \bs Dh} &
          {\bn N} & {\tt N}\\
{\bn t} & {\tt t} & {\bn z} & {\tt z} or {\tt \bs th} &
          {\bn d} & {\tt d} & {\bn x} & {\tt x} or {\tt \bs dh} &
          {\bn n} & {\tt n}\\
{\bn p} & {\tt p} & {\bn f} & {\tt f} or {\tt \bs ph} &
          {\bn b} & {\tt b} & {\bn \bh} & {\tt v} or {\tt \bs bh} &
          {\bn m} & {\tt m}\\
{\bn Y} & {\tt Y} or {\tt \bs j} & {\bn r} & {\tt r} &
          {\bn l} & {\tt l} & {\bn b} & {\tt b} &
          {\bn S} & {\tt S} or {\tt \bs sh} \\
{\bn F} & {\tt F} or {\tt \bs ss} & {\bn s} & {\tt s} &
          {\bn h} & {\tt h} & {\bn \rr} & {\tt R} or {\tt \bs rr} &
          {\bn \rh} & {\tt V} or {\tt \bs rh} \\
{\bn y} & {\tt y} & {\bn B} & {\tt B} or {\tt \bs t} &
          {\bn M} & {\tt M} or {\tt \bs anuswar} & {\bn H} & {\tt H} or {\tt \bs bisarga} &
          {\bn w} & {\tt w} or {\tt \bs chandra}\\
\hline
\end{tabular}}

\newsect{Some special symbols}
\begin{tabular}{cc|cc}
symbol & command & symbol & command \\ \hline
{\bn\gu} & {\tt gu} or {\tt\bs gu} & {\bn\ru} & {\tt ru} or {\tt\bs ru} \\
{\bn\rU} & {\tt rU} or {\tt\bs rU} or {\tt\bs roo} & {\bn \sh u} & {\tt \bs sh u} or {\tt\bs shu}\\
{\bn\hu} & {\tt hu} or {\tt\bs hu} & {\bn\hri} & {\tt h\bs r} or {\tt\bs hr} or {\tt\bs hri}\\
{\bn \jf} & {\tt \bs jf} or {\tt \bs jafala} &
     {\bn \rf} & {\tt \bs rf} or {\tt \bs rafala}\\
{\bn \Ref} & {\tt \bs Ref}\footnotemark & {\bn \hasanta} & {\tt \bs hasanta} or {\tt\bs hsn}\\
{\bn \bucks} & {\tt \bs bucks} or {\tt\bs taka} &
     {\bn \period} & {\tt \bs period} or {\tt\bs point} \\
{\bn ..} & {\tt ..} & & \\
\hline
\end{tabular}
\footnotetext{{\tt\bs ref} is a \LaTeX\ command.}

\noindent {\bf Note: } The vowel forms {\bn i , e, E} are to be put
before the consonant; the others are to be put after the consonant.
{\bn \o}-kar is produced by putting {\bn e}, the consonant and
then {\bn aa}. The same holds for {\bn e ou}. Ligature table is
used for the combinations {\tt aa, ee, ii, oo, uu, oi} and {\tt ou}. Similarly,
{\tt gu} prints {\bn gu} and not {\bn g{}u}. To get {\bn g{}u}, type
{\tt g\{\}u}. {\bn rU} is produced by {\tt rU}, whereas
{\tt roo} gives {\bn roo}. Finally, note that \hbox{\bn \jafala,\ \
\rafala, \Ref} and {\bn\ \hasanta} must all be typed after the
character to which they belong.

\newsect{Encoding of conjunct consonants}
The following conjunct consonants ({\bn \j u\kt aa\kkh r'}s) are defined.
These are the most common ones used in Bengali. The following
tables list for each conjunct consonant the component consonants,
the encoding and an example of
the conjunct consonant. In case a conjunct consonant has more than
one encoding, these codes are separated by commas. Conjunct consonants
having a ``jafala'', ``rafala'' or ``Ref'' are not listed. They can
be typeset by the commands {\tt \bs jafala, \bs rafala} etc. as described
in the last table. However, special combinations like {\bn \bhr} (which
is not the same as {\bn \bh{}\rf}) are treated separately and are found in
the following tables. (Also look at the section ``Ligatures''.)
\vfill \break
\centerline{\begin{tabular}[t]{cccl}\hline &&& \\
{\bn k + k} & {\bn \kk} & {\tt \bs kk} & {\bn i\dh\kk aar} \\
{\bn k + T} & {\bn \kT} & {\tt \bs kT} & {\bn \a e\kT aapaas}\\
{\bn k + t} & {\bn \kt} & {\tt \bs kt} & {\bn Daa\kt aar}\\
{\bn k + b} & {\bn \kw} & {\tt \bs kb,\bs kw} & {\bn p\kb}\\
{\bn k + m} & {\bn \km} & {\tt \bs km} & {\bn rui\km NI}\\
{\bn k + r} & {\bn \kr} & {\tt \bs kr} & {\bn b\kr}\\
{\bn k + l} & {\bn \kl} & {\tt \bs kl} & {\bn \kl Ib}\\
{\bn k + F} & {\bn \x} & {\tt \bs kF,\bs x,\bs kkh} & {\bn r\x A}\\
{\bn k + F + N} & {\bn \kkhN} & {\tt \bs kFN,\bs xN,\bs kkhN} & {\bn tee\xN}\\
{\bn k + F + m} & {\bn \kFm} & {\tt \bs kFm,\bs xm,\bs kkhm} & {\bn l\xm I}\\
{\bn k + s} & {\bn \ks} & {\tt \bs ks} & {\bn ir\ks A}\\
{\bn g + x} & {\bn \gdh} & {\tt \bs gdh,\bs gx} & {\bn mu\gdh}\\
{\bn g + n} & {\bn \gn} & {\tt \bs gn} & {\bn \a i\gn}\\
{\bn g + b} & {\bn \gb} & {\tt \bs gb} & {\bn idi\gb jy}\\
{\bn g + m} & {\bn \gm} & {\tt \bs gm} & {\bn \j u\gm}\\
{\bn g + l} & {\bn \gl} & {\tt \bs gl} & {\bn \gl Ain}\\
{\bn G + n} & {\bn \ghn} & {\tt \bs ghn,\bs Gn} & {\bn ib\Gn}\\
{\bn q + k} & {\bn \ngk} & {\tt \bs ngk,\bs qk} & {\bn Si\qk t}\\
{\bn q + k + F} & {\bn \ngkh} & {\tt \bs ngkh,\bs qkS} & {\bn \A kaa\qkS A}\\
{\bn q + K} & {\bn \nkh} & {\tt \bs nkh,\bs qK} & {\bn \sh\nkh}\\
{\bn q + g} & {\bn \ng} & {\tt \bs ng,\bs qg} & {\bn b\jf aa\ng}\\
{\bn q + G} & {\bn \ngh} & {\tt \bs ngh,\bs qG} & {\bn s\qG}\\
{\bn q + m} & {\bn \ngm} & {\tt \bs ngm,\bs qm} & {\bn baa\qm y}\\
{\bn c + c} & {\bn \cc} & {\tt \bs chch,\bs cc} & {\bn saa\cc aa}\\
{\bn c + C} & {\bn \chchh} & {\tt \bs chchh,\bs cC} & {\bn baa\chchh aa}\\
{\bn c + C + b} & {\bn \chchhb} & {\tt \bs chchhb,\bs chchhw,\bs cCb} & {\bn \u\chchhw aas}\\
{\bn c + Q} & {\bn \chn} & {\tt \bs chn,\bs cQ} & {\bn YA\chn A}\\
{\bn j + j} & {\bn \jj} & {\tt \bs jj} & {\bn l\jj aa}\\
{\bn j + j + b} & {\bn \jjb} & {\tt \bs jjb,\bs jjw} & {\bn \u\jjw l}\\
{\bn j + J} & {\bn \jjh} & {\tt \bs jjh,\bs jJ} & {\bn ku\jJ iTkaa}\\
{\bn j + Q} & {\bn \ggn} & {\tt \bs ggn,\bs jQ} & {\bn ib\ggn aan}\\
{\bn j + b} & {\bn \jb} & {\tt \bs jb,\bs jw} & {\bn \jb aalaa}\\
{\bn Q + c} & {\bn \nch} & {\tt \bs nch,\bs Qc} & {\bn m\Qc}\\
{\bn Q + C} & {\bn \nchh} & {\tt \bs nchh,\bs QC} & {\bn laa\QC nA}\\
{\bn Q + j} & {\bn \nj} & {\tt \bs nj,\bs Qj} & {\bn g\Qj}\\
{\bn Q + J} & {\bn \njh} & {\tt \bs njh,\bs QJ} & {\bn J\QJ A}\\
{\bn T + T} & {\bn \TT} & {\tt \bs TT} & {\bn \a\TT aailkA}\\
{\bn T + b} & {\bn \Tb} & {\tt \bs Tb,\bs Tw} & {\bn K\Tb aa}\\
{\bn D + D} & {\bn \DD} & {\tt \bs DD} & {\bn \u\DD een}\\
{\bn N + T} & {\bn \NT} & {\tt \bs NT} & {\bn k\NT ikt}\\
{\bn N + Z} & {\bn \NTh} & {\tt \bs NTh,\bs NZ} & {\bn l\NZ n}\\
{\bn N + D} & {\bn \ND} & {\tt \bs ND} & {\bn g\ND aar}\\
{\bn N + N} & {\bn \NN} & {\tt \bs NN} & {\bn ibF\NN}\\
{\bn N + m} & {\bn \Nm} & {\tt \bs Nm} & {\bn ihr\Nm y}\\
{\bn t + t} & {\bn \tt} & {\tt \bs tt} & {\bn \u\tt aal}\\
{\bn t + t + b} & {\bn \ttb} & {\tt \bs ttb,\bs ttw} & {\bn t\ttb}\\
{\bn t + z} & {\bn \tth} & {\tt \bs tth,\bs tz} & {\bn \a\shb\tz}\\
{\bn t + n} & {\bn \tn} & {\tt \bs tn} & {\bn r\tn}\\
{\bn t + b} & {\bn \tb} & {\tt \bs tb,\bs tw} & {\bn \tb rA}\\
&&& \\\hline
\end{tabular}
\hfill
\begin{tabular}[t]{cccl}\hline &&& \\
{\bn t + m} & {\bn \tm} & {\tt \bs tm} & {\bn \aa\tm A}\\
{\bn t + r} & {\bn \tr} & {\tt \bs tr} & {\bn sb\Ref\tr}\\
{\bn t + r + \u} & {\bn \tru} & {\tt \bs tru} & {\bn \tru iT}\\
{\bn z + b}& {\bn \bn \thb}& {\tt\bs thb,\bs thw,\bs zb}& {\bn p\ri\thw I}\\
{\bn d + g} & {\bn \dg} & {\tt \bs dg} & {\bn \u\dg AtA}\\
{\bn d + G} & {\bn \dgh} & {\tt \bs dgh,\bs dG} & {\bn \u\dG ATn}\\
{\bn d + d} & {\bn \dd} & {\tt \bs dd} & {\bn brA\dd}\\
{\bn d + x} & {\bn \ddh} & {\tt \bs ddh,\bs dx} & {\bn \u\dx Ar}\\
{\bn d + b} & {\bn \db} & {\tt \bs db,\bs dw} & {\bn \db eep}\\
{\bn d + v} & {\bn \dv} & {\tt \bs dv,\bs dbh} & {\bn s\dbh aab}\\
{\bn d + v + r} & {\bn \dvr} & {\tt \bs dvr,\bs dbhr} & {\bn \u\dbhr aa\nt}\\
{\bn d + m} & {\bn \dm} & {\tt \bs dm} & {\bn p\dm}\\
{\bn x + n} & {\bn \dhn} & {\tt \bs dhn,\bs xn} & {\bn g\r\xn}\\
{\bn x + b} & {\bn \dhb} & {\tt \bs dhb,\bs dhw,\bs xb} & {\bn \xb jaa}\\
{\bn n + T} & {\bn \nT} & {\tt \bs nT} & {\bn \A\nT Ak\Ref iTkA}\\
{\bn n + D} & {\bn \nD} & {\tt \bs nD} & {\bn b\jf A\nD}\\
{\bn n + t} & {\bn \nt} & {\tt \bs nt} & {\bn \a i\nt m}\\
{\bn n + t + \u} & {\bn \ntu} & {\tt \bs ntu} & {\bn j\ntu}\\
{\bn n + t + b} & {\bn \ntb} & {\tt \bs ntb,\bs ntw} & {\bn sA\ntb nA}\\
{\bn n + t + r} & {\bn \ntr} & {\tt \bs ntr} & {\bn Y\ntr}\\
{\bn n + z} & {\bn \nth} & {\tt \bs nth,\tt nz} & {\bn m\nz r}\\
{\bn n + d} & {\bn \nd} & {\tt \bs nd} & {\bn b\nd I}\\
{\bn n + d + b} & {\bn \ndb} & {\tt \bs ndb,\bs ndw} & {\bn\dw\ndw}\\
{\bn n + x} & {\bn \ndh} & {\tt \bs ndh,\bs nx} & {\bn g\nx}\\
{\bn n + n} & {\bn \nn} & {\tt \bs nn} & {\bn \a\nn}\\
{\bn n + b} & {\bn \nb} & {\tt \bs nb,\bs nw} & {\bn sm\nb y}\\
{\bn n + m} & {\bn \nm} & {\tt \bs nm} & {\bn j\nm}\\
{\bn n + s} & {\bn \ns} & {\tt \bs ns} & {\bn es\ns}\\
{\bn p + T} & {\bn \pT} & {\tt \bs pT} & {\bn ike\pT}\\
{\bn p + t} & {\bn \pt} & {\tt \bs pt} & {\bn s\pt Ah}\\
{\bn p + n} & {\bn \pn} & {\tt \bs pn} & {\bn \sb\pn}\\
{\bn p + p} & {\bn \pp} & {\tt \bs pp} & {\bn \dh aa\pp aa}\\
{\bn p + l} & {\bn \pl} & {\tt \bs pl} & {\bn \pl Abn}\\
{\bn p + s} & {\bn \ps} & {\tt \bs ps} & {\bn \a vI\ps A}\\
{\bn f + l} & {\bn \phl} & {\tt \bs phl,\bs fl} & {\bn i\fl paar}\\
{\bn b + j} & {\bn \bj} & {\tt \bs bj} & {\bn ku\bj}\\
{\bn b + d} & {\bn \bd} & {\tt \bs bd} & {\bn \sh\bd}\\
{\bn b + x} & {\bn \bdh} & {\tt \bs bdh,\bs bx} & {\bn \aa r\bx}\\
{\bn b + b} & {\bn \bb} & {\tt \bs bb} & {\bn maat\bb r}\\
{\bn b + l} & {\bn \bl} & {\tt \bs bl} & {\bn \bl iT\anuswar}\\
{\bn v + r} & {\bn \bhr} & {\tt \bs bhr,\bs vr} & {\bn \vr m}\\
{\bn v + l} & {\bn \bhl} & {\tt \bs bhl,\bs vl} & {\bn \vl Aidimr}\\
{\bn m + n} & {\bn \mn} & {\tt \bs mn} & {\bn in\mn}\\
{\bn m + p} & {\bn \mp} & {\tt \bs mp} & {\bn k\mp}\\
{\bn m + f} & {\bn \mph} & {\tt \bs mph,\bs mf} & {\bn l\mf}\\
{\bn m + b} & {\bn \mb} & {\tt \bs mb} & {\bn \a\mb l}\\
{\bn m + v} & {\bn \mbh} & {\tt \bs mbh, \bs mv} & {\bn d\mv}\\
{\bn m + v + r} & {\bn \mbhr} & {\tt \bs mbhr,\bs mvr} & {\bn s\mvr m}\\
{\bn m + m} & {\bn \mm} & {\tt \bs mm} & {\bn \A hA\mm k}\\
&&& \\\hline
\end{tabular}}

\noindent
\centerline{\begin{tabular}[t]{cccl}\hline &&& \\
{\bn m + l} & {\bn \ml} & {\tt \bs ml} & {\bn \ml aan}\\
{\bn l + k} & {\bn \lk} & {\tt \bs lk} & {\bn b\lk l}\\
{\bn l + g} & {\bn \lg} & {\tt \bs lg} & {\bn b\lg A}\\
{\bn l + g + \u} & {\bn \lgu} & {\tt \bs lgu} & {\bn fA\lgu n}\\
{\bn l + T} & {\bn \lT} & {\tt \bs lT} & {\bn p\lT n}\\
{\bn l + D} & {\bn \lD} & {\tt \bs lD} & {\bn ehA\lD aar}\\
{\bn l + p} & {\bn \lp} & {\tt \bs lp} & {\bn k\lp nA}\\
{\bn l + b} & {\bn \lb} & {\tt \bs lb,\bs lw} & {\bn ib\lb}\\
{\bn l + m} & {\bn \lm} & {\tt \bs lm} & {\bn gu\lm}\\
{\bn l + l} & {\bn \ll} & {\tt \bs ll} & {\bn p\rf\ph u\ll}\\
{\bn S + c} & {\bn \shch} & {\tt \bs shch,\bs Sc} & {\bn pun\Sc}\\
{\bn S + C} & {\bn \shchh} & {\tt \bs shchh,\bs SC} & {\bn iSre\SC d}\\
{\bn S + n} & {\bn \shn} & {\tt \bs shn,\bs Sn} & {\bn p\rafala\Sn}\\
{\bn S + m} & {\bn \shm} & {\tt \bs shm,\bs Sm} & {\bn ri\Sm}\\
{\bn S + l} & {\bn \shl} & {\tt \bs shl,\bs Sl} & {\bn \Sl aa\gh aa}\\
{\bn S + b} & {\bn \shb} & {\tt \bs shb,\bs shw,\bs Sb} & {\bn \Sb aapd}\\
{\bn F + k} & {\bn \ssk} & {\tt \bs ssk,\bs Fk} & {\bn pir\Fk aar}\\
{\bn F + k + r} & {\bn \sskr} & {\tt \bs sskr,\bs Fkr} & {\bn in\Fkr mN}\\
{\bn F + T} & {\bn \ssT} & {\tt \bs ssT,\bs FT} & {\bn k\FT}\\
{\bn F + Z} & {\bn \ssTh} & {\tt \bs ssTh,\bs FZ} & {\bn in\FZ ur}\\
{\bn F + N} & {\bn \ssN} & {\tt \bs ssN,\bs FN} & {\bn \u\FN}\\
{\bn F + p} & {\bn \ssp} & {\tt \bs ssp,\bs Fp} & {\bn bA\Fp}\\
{\bn F + f} & {\bn \ssph} & {\tt \bs ssph,\bs Ff} & {\bn in\Ff l}\\
&&& \\\hline
\end{tabular}
\hfill
\begin{tabular}[t]{cccl}\hline &&& \\
{\bn F + m} & {\bn \ssm} & {\tt \bs ssm,\bs Fm} & {\bn vI\Fm}\\
{\bn s + k} & {\bn \sk} & {\tt \bs sk} & {\bn pur\sk\r t}\\
{\bn s + k + r} & {\bn \skr} & {\tt \bs skr} & {\bn \skr een}\\
{\bn s + k + l} & {\bn \skl} & {\tt \bs skl} & {\bn e\skl raa}\\
{\bn s + K} & {\bn \skh} & {\tt \bs skh,\bs sK} & {\bn \sK ln}\\
{\bn s + T} & {\bn \sT} & {\tt \bs sT} & {\bn \sT pAr}\\
{\bn s + t} & {\bn \st} & {\tt \bs st} & {\bn \a\st}\\
{\bn s + t + \u} & {\bn \stu} & {\tt \bs stu} & {\bn b\stu}\\
{\bn s + t + r} & {\bn \str} & {\tt \bs str} & {\bn oi\str N}\\
{\bn s + z} & {\bn \sth} & {\tt \bs sth,\bs sz} & {\bn \u pi\sz t}\\
{\bn s + n} & {\bn \sn} & {\tt \bs sn} & {\bn \sn aan}\\
{\bn s + p} & {\bn \sp} & {\tt \bs sp} & {\bn pr\sp r}\\
{\bn s + p + l} & {\bn \spl} & {\tt \bs spl} & {\bn i\spl T}\\
{\bn s + f} & {\bn \sph} & {\tt \bs sph,\bs sf} & {\bn \A\sf aaln}\\
{\bn s + b} & {\bn \sb} & {\tt \bs sb,\bs sw} & {\bn \sb ed\sh}\\
{\bn s + m} & {\bn \sm} & {\tt \bs sm} & {\bn \sm rN}\\
{\bn s + l} & {\bn \dsl} & {\tt \bs dsl} & {\bn i\dsl p}\\
{\bn h + N} & {\bn \hN} & {\tt \bs hN} & {\bn apraa\hN}\\
{\bn h + n} & {\bn \hn} & {\tt \bs hn} & {\bn ici\hn t}\\
{\bn h + m} & {\bn \hm} & {\tt \bs hm} & {\bn b\rf\hm aa}\\
{\bn h + b} & {\bn \hb} & {\tt \bs hb,\bs hw} & {\bn \A\hb aan}\\
{\bn h + l} & {\bn \hl} & {\tt \bs hl} & {\bn \A\hl aad}\\
{\bn R + g} & {\bn \rrg} & {\tt \bs rrg,\bs Rg} & {\bn \kh\Rg}\\
&&& \\\hline
\end{tabular}}

\vfill

\newsect{Example 2}
\begin{tabular}{p{3in}p{3in}}
\noindent {\bf The input:}
\begin{verbatim}
{\bn
\begin{verse}
{\Large rbI\nd\rf nA\th\ \Th aakur}\\
{\large \sl geetibtaan}\\
\medskip
baaej kruN suer haay dooer \\
tb \ch rNtl\ch ui\mb t pA\nth bINA.\\
\e\ mm paa\nth i\ch t c\nch l \\
jaain naa kee \u e\dd e\sh.. \\
\j UzIg\ndh\ \a SA\nt\ smIer \\
xaay \u tlaa \u\chchhb Aes, \\
etmin ic\tt\ \u daasI er \\
indaaruN ibe\cC edr inSIez.. \\
\end{verse}
}
\end{verbatim}
&
\noindent {\bf The output:}
{\bn
\begin{verse}
{\Large rbI\nd\rf nA\th\ \Th aakur}\\
{\large \sl geetibtaan}\\
\medskip
baaej kruN suer haay dooer \\
tb \ch rNtl\ch ui\mb t pA\nth bINA.\\
\e\ mm paa\nth i\ch t c\nch l \\
jaain naa kee \u e\dd e\sh.. \\
\j UzIg\ndh\ \a SA\nt\ smIer \\
xaay \u tlaa \u\chchhb Aes, \\
etmin ic\tt\ \u daasI er \\
indaaruN ibe\cC edr inSIez.. \\
\end{verse}
}
\\
\end{tabular}

\newpage
\newsect{Changes from older versions}
In version 1.0 of \bwti, the Bengali alphabet was distributed among
different \METAFONT\ files. More specifically, the basic alphabet
(vowels, vowel forms, consonants, digits and punctuation symbols)
was defined in a \METAFONT\ source, while the conjunct consonants
were distributed among two different \METAFONT\ files.

After that I counted that the total number of glyphs turns out to be
less than 256. So I accumulated all the characters in a single font
file. Version 2.0 of \bwti\ is thus born. This change makes the Bengali
fonts more portable and helps better conversion to other formats.
In addition to the change in the font source, the style file is also
improved:
\begin{itemize}
\item The Bengali font is given a font family number so that one can
use them in math mode also. 
\item Commands are provided that allow one to write text that looks like
one typeset in boldface. ({\tt\bs bnbold, \bs bnBold} and {\tt\bs bnBOLD}).
\item Easy-to-use commands for typesetting headers of different sizes are
provided ({\tt\bs lbnh, \bs cbnh, \bs rbnh} etc.).
\end{itemize}

After that I was able to convert the fonts to some other formats (BDF,
GSF etc.). Some of these formats have some inhibited positions where one
should not define a glyph. As a result I had to change the encodings for
fonts in these other formats. This made the original font incompatible with
the converted ones. To restore compatiblity, version 3.0 of \bwti\ is
designed. Please note that the character encoding in \bwti\ 3.0 is a little
bit different from that in \bwti\ 2.0. This means that older files may
give erroneous outputs when compiled under version 3.0. I deeply regret
any inconvenience due to that. In addition to this change in encoding,
two new conjuncts are added and shapes of some of the old characters are
modified (beautified, perhaps) a little. Finally, like most other computer
modern fonts, all the digits have been designed to be of equal width.

Sorry for these rapid drastic changes. Hope to stabilize soon!!

\newsect{Address for correspondence}
I am here for your doubts, comments, suggestions and
criticisms. Feel free to communicate~---

\noindent
Dr.~Abhijit Das\\
Department of Mathematics\\
Indian Institute of Technology, Kanpur 208 016, INDIA\\
e-mail: {\tt ad\_rab\char`\@yahoo.com}\\
URL: {\tt http://in.geocities.com/ad\_rab/}

\newsect{Copyright information}
Copyright 1997--2002 by the author Abhijit Das {\tt[ad\_rab@yahoo.com]}

\noindent
{\tt bwti} is free software; you can redistribute it and/or modify
it under the terms of the GNU General Public License as published by
the Free Software Foundation; either version 2 of the License, or
(at your option) any later version.

\noindent
{\tt bwti} is distributed in the hope that it will be useful,
but WITHOUT ANY WARRANTY; without even the implied warranty of
MERCHANTABILITY or FITNESS FOR A PARTICULAR PURPOSE.  See the
GNU General Public License for more details.

\noindent
You should have received a copy of the GNU General Public License
along with {\tt bwti}; if not, write to the Free Software
Foundation, Inc., 59 Temple Place, Suite 330, Boston, MA  02111-1307  USA
\end{document}

%%% End of manual.tex
